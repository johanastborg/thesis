\chapter{Quantum Mechanics}

\section{Introduction}
Quantum mechanics represents a fundamental departure from the classical understanding of the physical world. Emerging in the early 20th century to address phenomena such as black-body radiation and the photoelectric effect, it introduced a probabilistic framework where physical states are described by vectors in a complex vector space.

While the initial formulation by Heisenberg (matrix mechanics) and Schrödinger (wave mechanics) appeared distinct, they were later unified. A rigorous mathematical foundation was subsequently established by John von Neumann, who formalized the theory using the language of Hilbert spaces and linear operators. This chapter explores the core principles of this formalism, setting the stage for the more advanced algebraic approaches discussed in later chapters. We begin by outlining the essential components of the theory, including the nature of observables, the role of measurement, and the dynamical evolution of quantum systems.
\section{Contents}
\begin{itemize}
    \item Type I - von Neumann Algebra
    \item Hilbert space
    \item Observables
    \item Postulates of Quantum Mechanics
    \item Density operator
\end{itemize}

\section{Type I - von Neumann Algebra}
A Type I von Neumann algebra corresponds to the algebra of all bounded linear operators $\mathcal{B}(\mathcal{H})$ on a Hilbert space $\mathcal{H}$, or to algebras that are isomorphic to direct sums of such algebras. This structure is foundational for standard quantum mechanics.
\begin{equation}
 \mathcal{M} = \mathcal{B}(\mathcal{H})
\end{equation}
\section{Hilbert space}
A Hilbert space $\mathcal{H}$ is a complete inner product space. It generalizes the notion of Euclidean space to infinite dimensions. The inner product $\langle \cdot | \cdot \rangle$ allows for the definition of angles and lengths.
\begin{equation}
 \langle \psi | \phi \rangle = \int \psi^*(x) \phi(x) dx
\end{equation}
\section{Observables}
Observables are physical quantities represented by self-adjoint (Hermitian) operators on $\mathcal{H}$. Their spectrum corresponds to the possible measurement values.
\begin{equation}
 \hat{A} = \hat{A}^\dagger
\end{equation}

\section{Postulates of Quantum Mechanics}
Postulates of Quantum Mechanics.

\subsection{Postulate 1}
The state of a physical system is described by a non-zero vector $|\psi\rangle$ in a complex Hilbert space $\mathcal{H}$.
\subsection{Postulate 2}
Physical quantities are represented by Hermitian operators $\hat{A}$ acting on $\mathcal{H}$.
\subsection{Postulate 3}
The possible results of a measurement of an observable $\hat{A}$ are its eigenvalues $a_n$.
\begin{equation}
    \hat{A} |n\rangle = a_n |n\rangle
\end{equation}
\subsection{Postulate 4}
The probability of obtaining the eigenvalue $a_n$ is given by the squared magnitude of the projection of the state onto the eigenstate $|n\rangle$.
\begin{equation}
    P(a_n) = |\langle n | \psi \rangle|^2
\end{equation}
\subsection{Postulate 5}
Immediately after a measurement yielding $a_n$, the state of the system collapses to the normalized eigenstate $|n\rangle$.
\begin{equation}
    |\psi\rangle \to \frac{\hat{P}_n |\psi\rangle}{\sqrt{\langle \psi | \hat{P}_n | \psi \rangle}}
\end{equation}
\subsection{Postulate 6}
The time evolution of a closed system is unitarily generated by the Hamiltonian operator $\hat{H}$.
\begin{equation}
    i\hbar \frac{d}{dt}|\psi(t)\rangle = \hat{H}|\psi(t)\rangle
\end{equation}

\section{Density operator}
The density operator $\rho$ describes the statistical state of a quantum system. It is a positive semi-definite operator with unit trace.
\begin{equation}
 \rho = \sum_i p_i | \psi_i \rangle \langle \psi_i |, \quad \text{Tr}(\rho) = 1
\end{equation}