\chapter{Gravitational Waves}

\section{Introduction}
This chapter explores Gravitational Waves.

\section{Gravitational waves as perturbation of linearized gravity}
This section discusses gravitational waves as a perturbation of linearized gravity.

The perturbed metric is given by:
\begin{equation}
    g_{\mu\nu} = \eta_{\mu\nu} + h_{\mu\nu}
\end{equation}
where $h_{\mu\nu} \ll 1$ represents the perturbation due to gravitational waves.

\section{Detection}
The detection of gravitational waves is primarily achieved through laser interferometry. Detectors such as LIGO (Laser Interferometer Gravitational-Wave Observatory) and Virgo operate as large-scale Michelson interferometers. These instruments measure the differential change in arm length caused by the passing gravitational wave.

The strain amplitude $h$ of a typical gravitational wave signal is extremely small, on the order of:
\begin{equation}
    h \approx \frac{\Delta L}{L} \sim 10^{-21}
\end{equation}
detecting such minute fluctuations requires isolating the test masses from thermal, seismic, and quantum noise. The sensitivity of these detectors is fundamentally limited by the quantum nature of light (shot noise and radiation pressure noise), an area where quantum measurement theory applies.
