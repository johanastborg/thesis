\chapter{Algebraic Quantum Field Theory}

\section{Introduction}
This chapter covers Algebraic Quantum Field Theory (AQFT).

\section{Modular Hamiltonian}
For a given density matrix $\rho$, the modular Hamiltonian $K$ is defined through the relationship:
\begin{equation}
    \rho = \frac{e^{-K}}{\text{Tr } e^{-K}}
\end{equation}
This expression resembles a thermal density matrix, where $K$ is analogous to a Hamiltonian, and the "temperature" is effectively set to 1.

\section{Axioms}
An algebraic quantum field theory is defined via a net of von Neumann algebras $\mathcal{A}(\mathcal{O})$ associated with open regions $\mathcal{O}$ in Minkowski spacetime, acting on a Hilbert space $\mathcal{H}$. These algebras satisfy the following axioms:

\subsection{Isotony}
The net of algebras preserves the inclusion relation of spacetime regions. If $\mathcal{O}_1 \subset \mathcal{O}_2$, then the corresponding algebras are nested:
\begin{equation}
    \mathcal{A}(\mathcal{O}_1) \subset \mathcal{A}(\mathcal{O}_2)
\end{equation}

\subsection{Locality (Einstein Causality)}
Observables in spacelike separated regions commute. If $\mathcal{O}_1$ and $\mathcal{O}_2$ are spacelike separated, then for all $A \in \mathcal{A}(\mathcal{O}_1)$ and $B \in \mathcal{A}(\mathcal{O}_2)$:
\begin{equation}
    [A, B] = 0
\end{equation}

\subsection{Poincaré Covariance}
There exists a strongly continuous unitary representation $U(\Lambda, a)$ of the Poincaré group $\mathcal{P}_+^\uparrow$ on $\mathcal{H}$ such that the net of observables transforms covariantly:
\begin{equation}
    U(\Lambda, a) \mathcal{A}(\mathcal{O}) U(\Lambda, a)^{-1} = \mathcal{A}(\Lambda \mathcal{O} + a)
\end{equation}

\subsection{Vacuum}
There exists a unique unit vector $\Omega \in \mathcal{H}$ (the vacuum state) which is invariant under the Poincaré translations $U(1, a)$:
\begin{equation}
    U(1, a) \Omega = \Omega
\end{equation}

\subsection{Spectrum Condition}
The spectrum of the energy-momentum generators $P_\mu$, derived from the translation subgroup, is confined to the closed forward light cone $\bar{V}_+$.
