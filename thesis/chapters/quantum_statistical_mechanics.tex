\chapter{Quantum Statistical Mechanics}

\section{Introduction}
This chapter discusses Quantum Statistical Mechanics. Modular theory generalizes the concept of temperature to quantum systems through Kubo-Martin-Schwinger (KMS) states.

\section{Density Operator}
The density operator $\rho$ describes the statistical state of a quantum system in a finite volume $V$. It is a positive operator with unit trace serving as the quantum analogue of the phase space probability distribution.
\begin{equation}
    \rho \ge 0, \quad \text{Tr}(\rho) = 1
\end{equation}

\section{States}
A state $\omega$ is defined as a positive linear functional on the algebra of observables $\mathcal{A}$. For a system described by a density matrix $\rho$, the expectation value of an observable $A$ is given by:
\begin{equation}
    \omega(A) = \text{Tr}(\rho A)
\end{equation}

\section{Type II - von Neumann Algebra}
Type II von Neumann algebras arise in the thermodynamic limit or when considering systems at infinite temperature. They possess a trace but do not have minimal projections (Type II$_1$) or are isomorphic to the tensor product of a Type II$_1$ factor and $\mathcal{B}(\mathcal{H})$ (Type II$_\infty$).

\section{Partition Function}
The partition function $Z$ quantifies the statistical properties of a system in the canonical ensemble at inverse temperature $\beta$:
\begin{equation}
    Z(\beta) = \text{Tr}(e^{-\beta H})
\end{equation}
where $H$ is the Hamiltonian of the system.

\section{Thermodynamic Limit}
The thermodynamic limit is defined as the limit where the volume $V \to \infty$ while the particle density $N/V$ remains constant. In this limit, the Hilbert space representation allows for unitarily inequivalent representations of the canonical commutation relations.

\section{Kubo-Martin-Schwinger (KMS) Condition}
The KMS condition characterizes equilibrium states at inverse temperature $\beta$ without relying on the trace operation, which may not be well-defined in the thermodynamic limit.

A state $\omega$ satisfies the KMS condition with respect to the time evolution automorphisms $\alpha_t$ if for any $A, B \in \mathcal{A}$, there exists a function $F(z)$ analytic in the strip $0 < \text{Im } z < \beta$ such that:
\begin{equation}
    F(t) = \omega(A \alpha_t(B)), \quad F(t + i\beta) = \omega(\alpha_t(B) A)
\end{equation}

The general form of the KMS condition for a quantum system is described by:
\begin{itemize}
    \item A von Neumann algebra $M$ of observables
    \item A one-parameter group of automorphisms $\alpha_t$ (time evolution)
    \item A state $\omega$ on $M$ (positive normalized linear functional) satisfying the analyticity condition above.
\end{itemize}
